\documentclass[letterpaper, 12pt, oneside, spanish]{tesis}

% Paquetes para idioma
\usepackage{babel}
\usepackage[utf8]{inputenc}
\usepackage[fixlanguage]{babelbib}

% Otros paquetes instalados
% Básicos
\usepackage{natbib}
\usepackage{enumerate}
\usepackage{datetime}
\usepackage{csquotes}
\longdate

% Para dibujar figuras
\usepackage{tikz}

% Para cambiar el color de las letras
\usepackage{color}

% Para incluir código (básico)
\usepackage{verbatim}
\usepackage{fancyvrb}

% Para incluir hipervínculos
\usepackage{hyperref}
\hypersetup{urlcolor=blue, colorlinks=false}

% Para más símbolos matemáticos
\usepackage{amsmath}
\usepackage{amsthm}
\usepackage{amssymb}

% Para colocar teoremas en cajas
\usepackage{mdframed}

% Para texto Lorem Ipsum
\usepackage{blindtext}

% Para suprimir algunas advertencias
\usepackage{silence}

\WarningFilter{blindtext}{}     % Supress the "spanish not defined, using English instead"

% Paquetes locales
% Puedes agregar paquetes locales (archivos .sty) en un subdirectorio % 'paquetes'.
% Utiliza la sintaxis \usepackage{paquetes/nombrePaquete}

% Todas las imágenes se cargan del subdirectorio 'img' por defecto.
\graphicspath{{img/}}

% Sangrías de 3 espacios (3 veces el espacio de la x)
\parindent 3ex

% Interlineado
\setlength{\baselineskip}{1.5pt}

% Interpárrafo
\setlength{\parskip}{16.5pt}

\topmargin 2cm

\renewcommand{\tablename}{Tabla}
\newcommand\listsymbolname{Acrónimos y Símbolos}

% Para colocar un margen a la izquierda
\newmdenv[
        hidealllines=true,
        leftline=true,topline=true,
        frametitleaboveskip=-5pt,
        linewidth=1.5pt,
    ]{leftframe}

\newenvironment{coloredframe}[2]{
    \mdfsetup{frametitle={\colorbox{white}{\space#1\space}},linecolor=#2}
    \begin{leftframe}
}{
    \end{leftframe}
}

% greens
\definecolor{caribbeangreen}{rgb}{0.0, 0.8, 0.6}

% blues
\definecolor{cerulean}{rgb}{0.0, 0.48, 0.65}
\definecolor{cyan process}{rgb}{0.0, 0.72, 0.92}

% reds
\definecolor{crimson}{rgb}{0.86, 0.08, 0.24}
\definecolor{darkcandyapplered}{rgb}{0.64, 0.0, 0.0}

\lstdefinelanguage{angler}{
    % add these letters as possible parts of keywords
    alsoletter={:, \\, =, -, >, \_, ., (, ), \{, \}},
    % list of keywords
    morekeywords={
        export, import, as, open, reopen, closed, with,
        where, let, in,
        forall, exists, select,
        behaviour, on, defines, is,
        operator, prefix, postfix, infixL, infixR, infixN
    },
    % symbol keywords
    morekeywords={
        [2]:, [2]=, [2]\\, [2]->, [2].,
        [2]\_, [2](, [2]), [2]\{, [2]\}
    },
    sensitive=true,                 % keywords are case-sensitive
    morecomment=[l]{\-\-},          % line comments
    morecomment=[s]{\{\-}{\-\}},    % block comments
    morestring=[b]",                % strings are in double quotes
    morestring=[d]',                % characters are in single quotes
}

\newcommand{\inlinemath}[1]{{\small\bfseries\color{cyan process}\texttt{$#1$}}}
\newcommand{\inlinecode}[1]{{\small\bfseries\color{darkcandyapplered}\texttt{#1}}}
\lstnewenvironment{anglercode}[1][]{
    \linespread{1}
    \lstset{
        language=angler,
        keywordstyle={\color{cerulean}\bfseries},   % style for keywords: bold
        keywordstyle={[2]\bfseries},
        commentstyle={\color{gray}},        % style for comments
        captionpos=t,                       % caption position: top
        frame=lt,                           % frame: left and top
        frameround=ffff,                    % roundness: TR, BR, BL, TL
        framerule=0.75pt,                   % thickness of the frame
        framesep=0.5cm,                     % separation from frame to code
        rulecolor=\color{gray},
        columns=flexible,                   % flexible character width
        keepspaces=true,                    % keep all spaces
        % margins
        aboveskip=1cm, xleftmargin=1.25cm, xrightmargin=1.25cm,
        % listings does not support UTF-8
        literate= {á}{{\'a}}1 {é}{{\'e}}1 {í}{{\'i}}1 {ó}{{\'o}}1 {ú}{{\'u}}1
                  {Á}{{\'A}}1 {É}{{\'E}}1 {Í}{{\'I}}1 {Ó}{{\'O}}1 {Ú}{{\'U}}1
                  {à}{{\`a}}1 {è}{{\`e}}1 {ì}{{\`i}}1 {ò}{{\`o}}1 {ù}{{\`u}}1
                  {À}{{\`A}}1 {È}{{\'E}}1 {Ì}{{\`I}}1 {Ò}{{\`O}}1 {Ù}{{\`U}}1
                  {ä}{{\"a}}1 {ë}{{\"e}}1 {ï}{{\"i}}1 {ö}{{\"o}}1 {ü}{{\"u}}1
                  {Ä}{{\"A}}1 {Ë}{{\"E}}1 {Ï}{{\"I}}1 {Ö}{{\"O}}1 {Ü}{{\"U}}1
                  {â}{{\^a}}1 {ê}{{\^e}}1 {î}{{\^i}}1 {ô}{{\^o}}1 {û}{{\^u}}1
                  {Â}{{\^A}}1 {Ê}{{\^E}}1 {Î}{{\^I}}1 {Ô}{{\^O}}1 {Û}{{\^U}}1
                  {œ}{{\oe}}1 {Œ}{{\OE}}1 {æ}{{\ae}}1 {Æ}{{\AE}}1 {ß}{{\ss}}1
                  {ű}{{\H{u}}}1 {Ű}{{\H{U}}}1 {ő}{{\H{o}}}1 {Ő}{{\H{O}}}1
                  {ç}{{\c c}}1 {Ç}{{\c C}}1 {ø}{{\o}}1 {å}{{\r a}}1 {Å}{{\r A}}1
                  {€}{{\EUR}}1 {£}{{\pounds}}1 {¿}{{\textquestiondown}}1
                  {¡}{{\textexclamdown}}1,
        #1                                  % settings for the lst environment
    }
}{}

% Macros
\newcommand{\projectTitle}{Angler}
\newcommand{\authorName}{Matteo José Ferrando Briceño}
\newcommand{\tutorName}{Ricardo Monascal}

\newdate{presentationdate}{16}{3}{2016}

\begin{titlepage}
    \title{
        \vspace{-2cm} \includegraphics[width=1.2in]{./usb.png} \\[.2cm]
        \large Universidad Simón Bolívar \\
        Decanato de Estudios Profesionales \\
        Coordinación de Ingeniería de la Computación
        \vfill \LARGE \projectTitle \vfill
    }
    \author{Por: \\
        \authorName \\[1.2cm]
        Realizado con la asesoría de: \\
        \tutorName \\[1.2cm]
        PROYECTO DE GRADO \\
        Presentado ante la Ilustre Universidad Simón Bolívar \\
        como requisito parcial para optar al título de \\
        Ingeniero de Computación
    }
    \date{Sartenejas, \monthname[\the\month] de \the\year}
\end{titlepage}

\begin{document}
\frontmatter
\maketitle
\setstretch{1.3}

% Se incluye el acta de evaluación, verificar que se corresponda
% con el formato aceptado actualmente por el Decanato.
% Pagina del acta final
\begin{titlepage}
\begin{center}

% Upper part
\includegraphics[scale=0.5]{usb.png} \\

\textsc {\large UNIVERSIDAD SIMÓN BOLÍVAR} \\
\textsc{DECANATO DE ESTUDIOS PROFESIONALES\\
COORDINACIÓN DE INGENIERÍA DE LA COMPUTACIÓN}

\bigskip
\bigskip
\bigskip
\bigskip
\bigskip
\bigskip

% Title
\textsc{ACTA FINAL PROYECTO DE GRADO}

\bigskip
\bigskip

\textsc{\textbf{\projectTitle}}

\bigskip
\bigskip
\bigskip
\bigskip

\begin{minipage}{\textwidth}
\centering
Presentado por: \\
\textsc{\textbf{\authorName}} \\

\bigskip
\bigskip
\bigskip

Este Proyecto de Grado ha sido aprobado por el siguiente jurado examinador: \\

\bigskip
\bigskip

{   % for \juryinfo to only exist here
\newcommand{\juryinfo}[1]{
\line(1,0){200} \\
#1 \\

\bigskip
\bigskip
}

% Despues de cada line coloca el (los) nombre(s) de
% cada uno de los integrantes del jurado.
\juryinfo{\tutorName}
\juryinfo{@jurado1}
\juryinfo{@jurado2}
}

\end{minipage}
\vfill

% Date/Fecha
{\large \bfseries Sartenejas, \displaydate{presentationdate}}

\end{center}
\end{titlepage}


% El resumen debe ser de una sola página
\addtotoc{Resumen}
\abstract{
\addtocontents{toc}{\vspace{1em}}
\blindtext

% Las palabras clave son generalmente los nombres de áreas de investigación a
% los cuales está asociado el trabajo. Generalmente son tres o cuatro.
\noindent \begin{small} \textbf{Palabras clave}: @palabra1, @palabra2, @palabra3.
\end{small}
}

% Iniciar nueva página luego del resumen
\clearpage
\setstretch{1.3}

% Agradecimientos
\acknowledgements{
\addtocontents{toc}{\vspace{1em}}
\blindtext
}
\clearpage

\pagestyle{fancy}

% Tabla de contenidos o índice
\lhead{\emph{Índice General}}
\tableofcontents

% Estos índices solamente se usan si el libro contiene figuras, tablas y
% algoritmos. Si alguno de estos no se utiliza, comentar o eliminar las líneas
% pertinentes.
\lhead{\emph{Índice de Figuras}}
\listoffigures

\lhead{\emph{Índice de Tablas}}
\renewcommand*\listtablename{Índice de Tablas}
\listoftables

%\lhead{\emph{Índice de Algoritmos}}
%\renewcommand*\listalgorithmname{Índice de algoritmos}
%\listofalgorithms

\setstretch{1.5}
\clearpage
\lhead{\emph{Acrónimos y símbolos}}
\listofsymbols{ll}
{

    % Aquí van las siglas
    \textbf{SIGLAS} & \textbf{S}iglas \textbf{I}sla \textbf{G}rafo
                      \textbf{L}aos \textbf{A}ve \textbf{S}erpiente\\
    \textbf{ACM} & \textbf{A}ssociation for \textbf{C}omputing \textbf{M}achinery\\
    &\\
    \hline
    &\\

    % Aquí van los símbolos
    $\iff$ & doble implicación, si y sólo si\\
    $\Rightarrow$ & implicación lógica\\
    $[u:=v]$ & sustitución textual de $u$ por $v$
}

%% ----------------------------------------------------------------
% End of the pre-able, contents and lists of things
% Begin the Dedication page

\setstretch{1.3}  % Return the line spacing back to 1.3

\pagestyle{empty}  % Page style needs to be empty for this page

\dedicatory{
    \textbf{Dedicatoria} \bigskip

    A @personasImportantes, por @razonesDedicatoria.
}

\addtocontents{toc}{\vspace{2em}}

\mainmatter
\pagestyle{fancy}

% Se incluye el cuerpo de la tesis en este documento.

\chapter*{Introducción}
\label{intro}
\lhead{\emph{Introducción}}
\addcontentsline{toc}{chapter}{Introducción}

% Descripción del problema, de lo general hacia lo específico
\blindtext

% Trabajos anteriores
\blindtext

En \cite{article:mixfix} se presenta un trabajo que \ldots

% \citeauthor{article:mixfix} es otro autor que \ldots

% Objetivo general
\blindtext

% Objetivos específicos
\blinditemize

Cachucha\footnote{Lorem Ipsum.}.

% Organización del trabajo
% Se describe brevemente qué se hace en cada capítulo
\blindtext[4]


% El número de capítulos varía. En mi libro fueron cuatro (sin contar
% introducción y conclusión).
\input{./partes/capitulo1}
% \renewcommand{\chapterTitle}{Marco Tecnológico\xspace}

\chapter{\chapterTitle}
\label{capitulo2}
\lhead{Capítulo 2. \emph{\chapterTitle}}

\Blindtext

% \def\chapterTitle{Marco Metodológico\xspace}

\chapter{\chapterTitle}
\label{capitulo3}
\lhead{Capítulo 3. \emph{\chapterTitle}}

\Blindtext

\renewcommand{\chapterTitle}{Desarrollo\xspace}

\chapter{\chapterTitle}
\label{capitulo4}
\lhead{Capítulo 4. \emph{\chapterTitle}}

\newenvironment{greenleft}
  {\begin{leftframe}[linecolor=caribbeangreen]}
  {\end{leftframe}}
\newenvironment{blueleft}
  {\begin{leftframe}[linecolor=cerulean]}
  {\end{leftframe}}
\newenvironment{redleft}
  {\begin{leftframe}[linecolor=crimson]}
  {\end{leftframe}}

\begin{greenleft}
    Investigación
\end{greenleft}

\begin{blueleft}
    Diseño
\end{blueleft}

\begin{redleft}
    Implementación
\end{redleft}

\section{Ideas iniciales}

Se comenzó con algunas ideas que consideramos básicas para el lenguaje.

\begin{itemize}
    \item El lenguaje debe ser explícito, es decir, no se deben inferir las ideas presentadas en un programa.
    \item La definición del lenguaje debe ser compacta.
    \item El lenguaje debe ser amistoso para principiantes.
    \item El sistema de tipos debe tener tipos dependientes.
    \item Los alcances del lenguaje deben ser determinados por la indentación.
\end{itemize}

\section{Expresiones, tipos y sus constructores}

\begin{blueleft}
Ya que el lenguaje debe tener tipos dependientes, se decidió desde un principio que un tipo es sencillamente una expresión, no un elemento especial del lenguaje, hace que en cualquier parte donde se espera un tipo, debería poder escribirse cualquier expresión, mientras que el chequeo de tipos pase sin errores.

Al hacer estos dos elementos uno solo, se definieron las diferentes construcciones posibles de expresiones.

\begin{itemize}
    \item Aplicación de funciones
    \item Parentización
    \item Lambda funciones
    \item Pattern matching
    \item Alcance de declaraciones para una expresión (\inlinecode{where}, \inlinecode{let-in})
    \item Cuantificadores universales y existenciales de tipos
\end{itemize}
\end{blueleft}

\section{Tipos y funciones, ¿abiertos o cerrados?}

\begin{blueleft}
Nos preguntamos porqué se ha hecho una diferencia tan clara en otros lenguajes de programación funcionales entre lo que es una función, un tipo y un constructor de tipo, así que comenzamos con la idea de que todos éstos fueran declarados de la misma manera.

\begin{anglercode}
Bool : Type
True : Bool
False : Bool

not : Bool -> Bool
not True = False
not False = True
\end{anglercode}

Ésta alternativa hacía que la única diferencia entre una función y un tipo es que no tenga declaraciones asociadas, pero el siguiente ejemplo nos dejó preguntándonos qué debería suceder:

\begin{anglercode}
Bool : Type
True : Bool
False : Bool

not : Bool -> Bool
not True = False
not False = True

True = False
\end{anglercode}

¿Ahora \inlinecode{True} deja de ser algo sobre lo que se puede hacer pattern matching, como en la función \inlinecode{not}?; como podemos ver, da paso a que un programa deje de funcionar con tan solo una línea sospechosa.

Otro ejemplo que da paso a problemas sería agregar un constructor a un tipo que no se debe:

\begin{anglercode}
Bool : Type
True : Bool
False : Bool

not : Bool -> Bool
not True = False
not False = True

IDK : Bool

not IDK     -- ¿qué debería reportarse aquí?
\end{anglercode}

Aquí agregamos un nuevo constructor a el tipo \inlinecode{Bool} y aunque tenga sentido usarlo en la función \inlinecode{not}, pues espera algo del tipo \inlinecode{Bool}, el programa va a dar un error al correr, no al ser analizado por errores estáticos.

Así que se nos ocurrieron los tipos cerrados, éstos son tipos cuyos constructores son listados inmediatamente en un alcance nuevo, y no se le puede agregar más constructores.

\begin{anglercode}
Bool : Type where
    True : Bool
    False : Bool

not : Bool -> Bool
not True = False
not False = True
\end{anglercode}

Y ahora ninguno de las confusiones anteriores suedería, al declarar el \inlinecode{IDK} no habría problema, pero al llamar \inlinecode{not IDK}, el intepretador reportaría un error estático.

Luego al ver \inlinecode{where} puede ser utilizado al final de cualquier expresión, por lo que no era una palabra que pudieramos utilizar para introducir los constructores, así que optamos por la palabra \inlinecode{with}. También el cambio muestra la diferencia semántica entre \inlinecode{where} y \inlinecode{with}, con el \inlinecode{where} se introduce un alcance sólo disponible en la expresión de éste, mientras que el \inlinecode{with} introduce elementos en el alcance exterior a éste.

\begin{anglercode}
Bool : t where t = Type
    with
        True : Bool
        False : Bool
\end{anglercode}

Pero parecía interensante la idea de tipos que sus constructores pudieran ser agregados mientras se fueran necesitando, viene en mente un ejemplo de monedas del mundo con tan solo algunas monedas \textit{base}, para que el usuario agregue las de interés.

\begin{anglercode}
Currency : Type
USD : Nat -> Currency
EUR : Nat -> Currency

toUSD : Currency -> Currency
toUSD (USD n) = USD n
toUSD (EUR n) = USD (n * 1.1)

-- ...

XBT : Nat -> Currency   -- bitcoin
toUSD (XBT n) = USD (n * 400)
\end{anglercode}

Así que si tenemos tipos abiertos y cerrados, tendremos también funciones abiertas y cerradas; ya que estamos buscando enlazar lo más posible estos dos conceptos:

\begin{anglercode}
not : Bool -> Bool where
    not True = False
    not False = True
\end{anglercode}

Pero los tipos abiertos seguían sufriendo de los problemas presentados al principio, habría que hacer una diferencia más explícita entre tipos y funciones.

Los tipos, abiertos y cerrados, deben indicar explícitamente cuáles son sus constructores de tipos para evitar el problema antes mencionado, de forma que llegamos a la siguiente sintaxis:

\begin{anglercode}
closed Bool : Type with
    True : Bool
    False : Bool

not : Bool -> Bool
not True = False
not False = True

closed Nat : Type with
    Z : Nat             -- cero
    S : Nat -> Nat      -- sucesor

open Currency : Type with
    USD : Nat -> Currency
    EUR : Nat -> Currency

toUSD : Currency -> Currency
toUSD (USD n) = USD n
toUSD (EUR n) = USD (n * 1.1)

-- ...

reopen Currency with
    XBT : Nat -> Currency

toUSD (XBT n) = USD (n * 400)
\end{anglercode}

Además todo este análisis nos hizo darnos cuenta de que aunque los tipos, constructores de tipos y funciones compartan muchas cualidades, no son lo mismo, se diferencian semánticamente y ésto debería mostrarse en su definición también.

Haciendo finalmente claras las diferencias, dejamos de un lado la idea de funciones cerradas; aunque éstas pueden ser teóricamente interesantes, no conseguimos gran utilidad práctica, ya que de querer \textit{cerrar} una función abierta, simplemente se deben escribir todos los casos posibles de sus argumentos.

\begin{anglercode}
isUSD : Currency -> Bool
isUSD (USD _) = True
isUSD _ = False         -- *cierra* la función

-- ...

isUSD (XBT _) = True    -- este caso nunca correrá
\end{anglercode}

\end{blueleft}


\chapter{Conclusiones y Recomendaciones}
\label{ch:conclusions}
\lhead{\emph{\nameref{ch:conclusions}}}

\Blindtext

% Incluir recomendaciones para trabajos futuros
\blindenumerate


% El estilo de la bibliografía es AAAI, definido en el archivo aaai.bst.
\label{Bibliography}
\bibliography{bibliografia}
\lhead{\emph{Bibliografía}}
\bibliographystyle{alpha}
\addtocontents{toc}{\vspace{2em}}

% Apéndices
\appendix
\input{./partes/apendiceA}
\chapter{@nombreApendice}
\label{ad:B}
\lhead{Apéndice B. \emph{\nameref{ad:A}}}

\Blindtext

\addtocontents{toc}{\vspace{2em}}

\backmatter

\end{document}
