\chapter{@nombreApendice}
\label{apendiceA}
\lhead{Apéndice A. \emph{@nombreApendice}}

% En los apéndices se incluye cualquier información que no sea esencial para la
% comprensión básica del trabajo, pero provea ejemplos y casos de estudio
% extendidos que permitan un análisis más exhaustivo.

\section{@sección}
\blindtext

\subsection{@subsección}
\Blindtext

``Saludo''.

\begin{figure}[h!]
\centering
\includegraphics[width=0.5\textwidth]{operator-graph.pdf}
\caption[Grafo]{Grafo gris.}
\label{imagen:grafo}
\end{figure}

\begin{figure}[h!]
\centering
\includegraphics[width=\textwidth]{operator-graph.pdf}
\caption[Grafo coloreado (esto sale en la tabla de contenidos)]{Grafo con color.}
\label{imagen:grafodecolores}
\end{figure}
